\newpage
\chapter{Introduction}
\paragraph{}Today the Automated guided vehicle plays an important role in the design of new factories and warehouses. In an automated process, AGVs are programmed to communicate with other robots to ensure the product is moved smoothly through the warehouse, whether it is being stored for future use or sent directly to shipping areas.  AGV is one kind of transportation that follows the given respective paths and route. It is widely used industrial fields and community services as well as in dangerous working it is able to sense and respond in the given environment. The first AGV was brought to market in the 1950s, by Barrett Electronics of Northbrook, and at the time it was simply a tow truck that followed a wire in the floor instead of a rail. over the years the technology has become more sophisticated and today automated vehicles are mainly Laser navigated ex: LGV which works more accurately and precisely in industrial fields, manufacturing processes medical fields. The robot perform hard, dangerous and accurate work in order to make our life easy as they can work for hours without taking rest. They can work without making errors in very less time. The AGV can also store objects on a bed. The objects can be placed on a set of the conveyor and then pushed off by reversing them. Some AGVs use forklifts to lift objects for storage. AGVs are employed in nearly every industry, including, pulp, paper, metals, newspaper, and general manufacturing. Transporting materials such as food, linen or medicine in hospitals is also done. An AGV can also be called a laser guided vehicle (LGV) or self-guided vehicle (SGV). Lower cost versions of AGVs are often called Automated Guided Carts (AGCs) and are usually guided by magnetic tape.

\section{Types of navigation in AGVs}
\subsection{Wired Navigation}
\paragraph{}A slot is cut in to the floor and a wire is placed approximately 1 inch below the surface. This slot is cut along the path the AGV is to follow. This wire is used to transmit a radio signal. A sensor is installed on the bottom of the AGV close to the ground. The sensor detects the relative position of the radio signal being transmitted from the wire. This information is used to regulate the steering circuit, making the AGV follow the wire.

\subsection{Guide tape navigation}
\paragraph{}AGVs use tape for the guide path. The AGV is fitted with the appropriate guide sensor to follow the path of the tape. One the major advantage of tape over wired guidance is that can be easily removed and relocated if the course needs to change. Colored tape is initially less expensive, but lacks the advantage of being embedded in high traffic areas where the tape may become damaged or dirty. The flexible magnetic bar can also be embedded in the floor like wire but works under the same provision as magnetic tape and so remains unpowered or passive. Another advantage of magnetic guide tape is the dual polarity.

\subsection{Laser target navigation}
\paragraph{}The navigation is done by mounting reflective tape on walls, poles or fixed machines. The AGV carries a laser transmitter and receiver on a rotating turret. The laser is transmitted and received by the same sensor. The angle and (sometimes) distance to any reflectors that in line of sight and in range are automatically calculated. This information is compared to the map of the reflector layout stored in the AGV's memory. This allows the navigation system to triangulate the current position of the AGV. The current position is compared to the path programmed in to the reflector layout map. The steering is adjusted accordingly to keep the AGV on track. It can then navigate to a desired target using the constantly updating position