\chapter{Experimental Analysis}
\section{Manual Control}
\paragraph{} We have tested our robot for manual control.


\begin{table}[htbp]
\caption{Manual Control Results}
\begin{center}
\begin{tabular}{|p{1.5cm}|p{6cm}|p{4cm}|}
\hline \textbf{Sr. No.} & \textbf{Parameter} & \textbf{Details}\\
\hline 1 & Max Speed & 250 rpm or 1.5 m/s\\
\hline 2 & Weight Carrying Capacity & 5 kg\\
\hline 3 & Average Battery Time & 3 hour \\
\hline 4 & Motion of Robot & 360 degree\\
\hline 5 & Head Control Error & 1-2 degree\\
\hline
\end{tabular}
\end{center}
\end{table}

\section{Machine Vision}
\paragraph{} Some of the key points of machine vision.

\begin{table}[htbp]
\caption{Machine Vision Results}
\begin{center}
\begin{tabular}{|p{1.5cm}|p{6cm}|p{4cm}|}
\hline \textbf{Sr. No.} & \textbf{Parameter} & \textbf{Details}\\
\hline 1 & No. of task at a time & Line + Angle + QR \\
\hline 2 & Average Frames per second & 16 frames per sec\\
\hline 3 & Frame Resolution & 640 * 480 pixels\\
\hline 4 & Code Execution Time &  0.0625 sec\\
\hline 5 & Accuracy of Completed Run & 80 \%\\
\hline
\end{tabular}
\end{center}
\end{table}

* Reducing frame resolution increases frames per second 

\section{Robotic Arm}
\paragraph{} Some of the results related to robotic arm is shown below:

\begin{table}[htbp]
\caption{Robotic Arm Results}
\begin{center}
\begin{tabular}{|p{1.5cm}|p{6cm}|p{4cm}|}
\hline \textbf{Sr. No.} & \textbf{Parameter} & \textbf{Details}\\
\hline 1 & Degree of Freedom & 2 \\
\hline 2 & No. of actuator & 1\\
\hline 3 & Weight holding capacity & 400 gm\\
\hline 4 & Average time for Pick or Place &  2 sec\\
\hline 5 & Accuracy & 60 \%\\
\hline
\end{tabular}
\end{center}
\end{table}

* More testing and robustness needed.

\section{Path Planning}
\paragraph{} Key parameter mentioned below:

\begin{table}[htbp]
\caption{Path Planning Results}
\begin{center}
\begin{tabular}{|p{1.5cm}|p{6cm}|p{4cm}|}
\hline \textbf{Sr. No.} & \textbf{Parameter} & \textbf{Details}\\
\hline 1 & Type of Node & QR code \\
\hline 2 & No. of Node & 6 \\
\hline 3 & Algorithm used & A* \\
\hline 4 & Mode of Acquiring Start \& End & GUI\\
\hline 5 & Default Start and End & Available\\
\hline
\end{tabular}
\end{center}
\end{table}

* Iteration of Path Planning and Robot remaining.

\newpage

\section{Graphical User Interface (GUI)}
\paragraph{} Points concerned with GUI shown below:

\begin{table}[htbp]
\caption{Graphical User Interface (GUI) Results}
\begin{center}
\begin{tabular}{|p{1.5cm}|p{6cm}|p{4cm}|}
\hline \textbf{Sr. No.} & \textbf{Parameter} & \textbf{Details}\\
\hline 1 & Languages used & HTML, Javascript \\
\hline 2 & Controller & ESP8266 \\
\hline 3 & Protocol & TCP \\
\hline 4 & No. of pages & 2\\
\hline 5 & Password Protected & Yes\\
\hline
\end{tabular}
\end{center}
\end{table}

* This type of GUI are very useful in industries for visualization purpose.

\section{Low Voltage Detection}
\paragraph{} Details of Low Voltage Detection convered below:

\begin{table}[htbp]
\caption{Low Voltage Detection Results}
\begin{center}
\begin{tabular}{|p{1.5cm}|p{6cm}|p{4cm}|}
\hline \textbf{Sr. No.} & \textbf{Parameter} & \textbf{Details}\\
\hline 1 & Voltage Displayed & Till 2 decimal\\
\hline 2 & Accuracy & 95 \% \\
\hline 3 & No. of cells & 3 \\
\hline 4 & Mode of Alert & Buzzer\\
\hline
\end{tabular}
\end{center}
\end{table}

* Self Charging Feature can be added in future with it.

