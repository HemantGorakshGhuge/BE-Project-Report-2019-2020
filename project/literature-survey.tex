\chapter{Literature Survey}

\section*{2.1 Vehicle pose estimation algorithm for parking automated guided vehicle\cite{ZNing}}
\subsection*{Authors: Zhixiong Ning, Xin Wang, Jun Wang, Huafeng Wen}
\subsection*{Published in: International Journal of Advanced Robotic Systems, pp. 1-11, 2020}
\paragraph{}This paper proposed a method which is based on symmetry of wheel point clouds collected by 3D lidar. This method include cell based method and support vector machine classifier to segment ground point clouds. The symmetry of wheel point clouds and obstacle point clouds is estimated by iterative closet algorithm. After estimation of vehicle pose, this method is compared with registration method for combination of sample consensus initial alignment and iterative closet point algorithm. This method simplifies the training data and the results of training data and the results of ground segmentation is more accurate. This method can also save time as it uses registration method. Experimental results of this method concludes that it is feasible and high precision method.
\newpage
\section*{2.2 Resource management in decentralized industrial Automated Guided Vehicle systems\cite{KShariatmadar}}
\subsection*{Authors: M. De Ryck, M. Versteyhe, K. Shariatmadar}
\subsection*{Published in: Journal of Manufacturing Systems, Vol. 54, pp. 204-214, 2020}
\paragraph{}In this paper,an advanced method is proposed where AGV choose whether it has to charge before or after performance of task and how much time it will take for charging. Using this method, transportation and resource management can be handled efficiently. This paper includes sections like battery management in which performance of battery can be improved, and explanation of  TSP which helps to model real world problems and to test optimal algorithms. This paper concludes that the AGV will optimally include charging stations and the time it required for charging, also the battery consumption is depended on factors like speed, weight, acceleration, friction etc.

\section*{2.3 Velocity Control of Omni Drive Robot using PID Controller and Dual Feedback\cite{ABirari}}
\subsection*{Authors: Akshay Birari, Amit Kharat, Purva Joshi, Rahul Pakhare, Unmesh Datar, Vaibhav Khotre}
\subsection*{Published in: IEEE First International Conference on Control, Measurement and Instrumentation (CMI), pp. 295-299, 2016}
\paragraph{}This paper explains the navigation of omni drive robot with velocity control and their equations. This robot uses two feedback systems i.e rotary encoder and laser distance sensors. Using mathematical equations rotary encoder and PID controller calculates accurate velocity. While laser distance sensors provide real time position of robot to manage its trajectory. Experimental results shows that this is the first version of robot which can navigate through complex workspace autonomously within acceptable error limits.

\section*{2.4 Building a Warehouse Control System using RIDE\cite{JLopez}}
\subsection*{Authors: Joaquin L\'{o}pez, Diego P\'{e}rez, Iago Vaamonde, Enrique Paz, Alba Vaamonde, Jorge Cabaleiro}
\subsection*{Published in: Robot 2015: Second Iberian Robotics Conference - Advances in Robotics, Lisbon, Portugal, 19-21 November 2015, Volume 2, pp. 757–768, 2015}
\paragraph{}There is a growing interest in the use of Autonomous Guided Vehicles (AGVs) in the warehouse control systems (WCS) in order to avoid installing fixed structure that complicate and reduce the flexibility to future changes. In this paper a highly flexible and hybrid operated WCS, developed using Robotics Integrated Development Environment (RIDE), is presented. The prototype is a forklift with cognitive capabilities that can be operated manually or autonomously and it is now being tested in a warehouse located in the Parque Technológico Logístico (PTL) of Vigo.

\section*{2.5 Implementation and Evaluation of Image Processing on Vision Navigation Line Follower Robot\cite{WElhady}}
\subsection*{Authors: Walaa E. Elhady, Heba A. Elnemr, Gamal Selim}
\subsection*{Published in: Journal of Computer Science, pp. 1036-1044, 2014}
\paragraph{}This paper purposes a line follower robot with computer vision and MATLAB platform. For computer vision, robot uses Lee and Khan filters along with basic morphological operations. The results of this system are evaluated from the points of PSNR, entropy and image smoothness. The proposed LFR is robust against environmental factors as well as line colour and can be extended further to follow coloured objects.

\section*{2.6 Vision Sensor-Based Driving Algorithm for Indoor Automatic Guided Vehicles\cite{QVan}}
\subsection*{Authors: Quan Nguyen Van, Hyuk-Min Eum, Jeisung Lee, Chang-Ho Hyun}
\subsection*{Published in: International Journal of Fuzzy Logic and Intelligent Systems, Vol. 13, No. 2, pp. 140-146, 2013}
\paragraph{}In this paper, a vision sensor based driving algorithm is proposed using two cameras. One camera is mounted to observe the environment and to detect markers. The second camera is used is used to compensate the distance between the wheels and markers. The experimental results shows that this method worked perfectly using two low cost USB cameras. This control system does not require the programmed destination target. Thus AVG can operate in flexible manner using layout.

\section*{2.7 Wireless Control of an Automated Guided Vehicle\cite{RPiyare}}
\subsection*{Authors: Rajeev K Piyare, Ravinesh Singh}
\subsection*{Published in: Proceedings of the International Multi-Conference of Engineers and Computer Scientists 2011 Vol II, IMECS 2011, pp. 828-833, 2011}
\paragraph{}This paper represents the design, implementation, and experimental results of a radio frequency (RF) based wireless control of a distributed Peripheral Interface Controller (PIC) micro-controller based Automated Guided Vehicle (AGV), which is known as ROVER II (Roaming Vehicle for Entity Relocation). ROVER II was designed in-house as general purpose guide path following mobile platform for material handling and transportation within a manufacturing facility.

\section*{2.8 Camera-based AGV Navigation System for Indoor Environment with Occlusion Condition\cite{NIsozaki}}
\subsection*{Authors: Naoya Isozaki, Daisuke Chugo, Sho Yokota, Kunikatsu Takas}
\subsection*{Published in: Proceedings of the 2011 IEEE International Conference on Mechatronics and Automation, pp. 778-783, 2001}
\paragraph{}In our current research, we are developing a practical mobile vehicle navigation system which is capable of controlling multiple vehicle on the general environment. In order to release the practical use, navigation system should have high accuracy and be low cost. Therefore, in this paper, authors proposed the vehicle navigation system which realizes navigation using high accuracy localization scheme by ceiling camera with infrared filters and LED markers. One topic is system integration which consists of high accuracy recognition system for LED marker and AGV controller with low cost. The other topic is novel AGV navigation scheme with high accuracy under occlusion condition. For practical use, mobile vehicle are required to continue its task with safety and high accuracy on temporary occlusion condition. The developed system can navigate target vehicle based on estimation error with simple algorithm.
\vspace{-0.3cm}
\section*{2.9 Effective Real-Time Wireless Control of an Autonomous Guided Vehicle\cite{CLozoya}}
\subsection*{Authors: C. Lozoya, P. Martí, M. Velasco, J.M. Fuertes}
\subsection*{Published in: 2007 IEEE International Symposium on Industrial Electronics, pp. 2876-2881, 2007}
\paragraph{}Wireless communication systems used in industrial environments must guarantee that the information is sent and received within precise time-bounds. However the nature of the radio channels and the medium access control (MAC) generates random communication delays. For networked control systems, these delays can cause severe performance problems. This paper presents an autonomous guide vehicle (AGV) path tracking control system whose mitigate the negative effects of delays, authors propose a Kalman-based network delay estimator in the controller that provides effective control performance. Here the approach is compared to previously propose statistical estimation algorithms by evaluating the vehicle’s travelling time and path deviation.

\section*{2.10 Survey of research in the design and control of automated guided systems\cite{Iris}}
\subsection*{Author: Iris F. A., Vis}
\subsection*{Published in: European Journal of Operational Research, Volume 170, Issue 3, pp. 677-709, 2006}
\paragraph{} Automated Guided Vehicles (AGVs) are used for the internal and external transport of material. Traditionally, AGVs were mostly used at manufacturing systems. Currently AGVs are used for repeating for transportation tasks in the other areas such as warehouse, container terminal and external (underground) transportation systems. This paper discusses literature related to design and control issues of AGV systems at manufacturing, distribution, transshipment and transportation systems. It is concluded that most model can be applied for design problem at manufacturing centers. Some of these models and new models already proved to successful in large AGV systems. In fact, new analytical and simulation models need to developed for large AGV systems to overcome large computational time, NP- completeness, congestion, deadlock and delays in system and finite planning horizons. Authors specify more specific research perspectives in the design and control of AGV systems in distribution, transshipment and transportation systems.
\newpage

\section*{2.11 Semi-guided navigation of AGV though iterative learning\cite{TFujimoto}}
\subsection*{Author: T. Fujimoto, J. Ota, T. Arai, T. Ueyama, T. Nishiyama}
\subsection*{Published in: Proceedings 2001 IEEE/RSJ International Conference on Intelligent Robots and Systems, pp. 968-973, 2001}
\paragraph{}In this paper, the authors aim at realizing an accurate navigation system of automated guided vehicles (AGV). The authors propose a way of estimation positioning error with magnetic tape, which is widely used in a factory as an external sensor. However, flexibility of path relocation is insufficient because, in general, the tape should be laid down on the floor from a start point to goal point so that AGV can reach their target. To overcome inefficiency, the authors firstly proposed semi-guided navigation methodology by means of two kinds of magnetic tapes based on error analysis. The semi-guided navigation means that magnetic tape are only placed at the start and goal points individually. Therefore, this system enables us to remove most of the magnetic tape. Moreover, the authors attempt a fixed model learning to prevent stationary error while AGV runs iteratively. Finally, the authors carry out experiments to evaluate the efficiency of the proposed method.


